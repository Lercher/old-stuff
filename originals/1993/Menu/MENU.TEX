%%   format=glplain

\documentstyle[12pt, german, twocolumn]{article} 
% Specify your document style here
%
% LaTeX source.
%
%  Note: you should define (or \input) your macros and preamble 
%  declarations here before the \begin{document} statement.
%      
%
%  Define (or \input) your macros and preamble declarations here...
%
%
\parindent = 0pt
\parskip = 3pt plus 1pt
\hfuzz=3pt

\addtolength\topmargin{-27pt}
\addtolength\textheight{68pt}

\title{Menu - Kurzbeschreibung}
\author{Dipl.{} Math.{} Martin Lercher}
\date{Version 1.5 vom M"arz 1993\\gesetzt \today}
\begin{document}
%
% The <start-text> tag separates the macro definitions
% from the body of the document.  You must have exactly one
% occurrence of this tag in your document (if you need to move it,
% you can use the File--Options... panel to disable Context rules,
% then cut and paste the tag, and finally re-enable Context rules).
%
\def\<#1>{{\verb|<|\it #1\/\verb|>|}}

\onecolumn
\maketitle{}

%
% Enter the body of your document from here down.
%
% Your file should consist of valid LaTeX input; furthermore,
% you can use the Create menu (off the File button) to insert
% Publisher tables, equations, and graphics into your document.
%
%
%%%%%%%%%%%%%%%%%%%%%%%%%%%%%%%%%%%%%%%%%%%%%%%%%%%%%%%%%%%%%%%%%%%%%
\section{Allgemeine Beschreibung des Mini-Menu-Systems.}

Das Mini-Menu-System ist ein Programm f\"ur PC-kompatible Rechner,
um durch {\it Kommandos\/} in einer {\it Menu-Datei\/}
frei konfigurierbare, hierarchische Menus anzuzeigen.

Die Hauptaufgabe des Systems
besteht darin, einen unerfahrenen Benutzer vor der
rauhen MSDOS-Welt (allgemein oder vor speziellen Programmen) 
zu besch\"utzen. Es l\"a\ss t ihn mittels einzelner
wohl\-er\-kl\"ar\-ter Tastendr\"ucke 
f\"ur ihn unsichtbare DOS-Kommandos
ausf\"uhren. Na\-t\"ur\-lich geht das nicht von alleine, er
braucht schon jemanden, der dieses Menu-System programmiert.

Das System ist ausgelegt f\"ur 26 verschiedene Menus mit Namen 
{\tt A-Z}, es verwaltet 26
Variable, denen sogenannte Picklisten (das ist eine Liste der 
letzten Eingaben der Variable) zugeordnet werden k\"onnen.
Ferner gibt es zehn Selektoren mit Namen {\tt 0-9}, denen Werte von Null
bis 20 zugewiesen werden k"onnen.
$$
   \hbox{Aufruf: {\tt menu $[$-k$]$}[\<Menu-name.ext>]}
$$
Falls keine Menu-Datei \<Menu-name.ext> angegeben wird,
wird die Datei
{\tt menu} daf\"ur benutzt. Die angegebene Datei wird zuerst
im aktuellen Directory und dann bei Nichterfolg in dem Directory
gesucht, in dem sich das Programm {\tt menu.exe} befindet.
Ist die Option {\tt -k} (key) gesetzt, so startet das Programm nur,
wenn w"arenddessen eine Taste gedr"uckt wurde (z.B. in einer Batchdatei).
Diese Taste wird dem ersten Menu durchgereicht.

Danach wird noch eine eventuell vorhandene Datei
$$\hbox{{\it Menu-name}{\tt .sav}}$$
aus dem aktuellen Directory geladen.
Als n\"achstes erfolgt der Aufruf des
Menus~{\tt A}, was den Start der Hauptschleife bedeutet. Wird dieses
Menu beendet, so endet auch das Mini-Menu-System.
Ein Menu besteht aus einem Text, der angezeigt wird, und einer Liste
von Tastencodes, denen Befehle zugeordnet sind. Beim Aufruf eines 
Menus wird der Text umrahmt ausgegebeben, in der untersten Zeile
die Eingabeauf{}forderung und rechts unten Datum und Uhrzeit.
Anschlie\ss end wird auf einen Tastendruck gewartet, der den/die 
zugeordneten {\it Befehle} in der Reihenfolge der Liste zur 
Ausf\"uhrung bringt. (Vgl. 3) Unabh\"angig davon k\"onnen die
zehn Funktionstasten {\tt F1-F10} mit je einem Befehl belegt
werden, der von jedem Menu aus durch die entsprechende Funktionstaste
aufgerufen werden kann.

\section{Beschreibung der Kommandos.}

Kommandos werden in der Menu-Datei angegeben und werden aus einem 
{\it Kommandonamen\/} bestehend aus einem Zeichen und optionalen 
{\it Parametern\/} zusammengesetzt, die 
ohne Abstand an den {\it Kommandonamen\/} 
anschlie\ss en. Ein Variablenname $V$ ist ein Buchstabe zwischen
{\tt A} und {\tt Z}. Eine {\it Pickliste\/}
 ist eine durch~$^\circ$ getrennte
Liste von Zeichenfolgen. Der Begriff {\it Befehl\/}
 wird im n\"achsten
Abschnitt erl\"autert, ist im Prinzip aber nur eine Zeichenfolge.



Hier die Liste der Kommandos:

\def\ii#1(#2){{\verb|#1|\global\def\p{\<#2>}\p}&}
\def\io#1[#2]{{\verb|#1|\global\def\p{\<#2>}\rm[\p]}&}
\def\iii#1#2(#3){{\verb|#1|$#2$\global\def\p{\<#3>}\p}&}
\def\iko#1#2[#3]{{\verb|#1|$#2$\global\def\p{\<#3>}\rm[\p]}&}
\def\ik#1#2{{\verb|#1|$#2$}&}
\def\i#1{{\verb|#1|}&}

\begin{tabular}{lp{10,5true cm}}
\tolerance=9900
\ii !(Text)Zeige den \p\ w\"ahrend des Einlesens der Menu-Datei an.\\
\ii :(prompt)Definiere die Eingabeauf{}forderung in der letzten Zeile
  durch \p. Voreingestellt ist ``{\tt Bitte w\"ahlen:}''. 
  Es erfolgt Variablenersetzung bei der Anzeige.\\

\iii \$V(Inhalt)Lade die Variable $V$ mit \p\ als Inhalt.\\
\iii \%V(Liste)Ordne der Variablen $V$ die Pickliste \p\ zu.\\
\ii \#(Kommentar)Die gesamte Zeile wird ignoriert.\\
\ii {\rm[\tt 0\rm-\tt9\rm]}(Befehl)Lege die Befehle f\"ur
  die Funktionstasten {\tt F1-F(1)0} fest. Davon mittels \verb|m|$M$
  aufgerufene Menus sollten jedoch stets mit \verb|\iq| als Befehl enden.
  \\
\i {m$M$}Definiere Menu $M$.\\
\i {g$M$}Starte mit Menu $M$. Default ist \verb|A|.
\end{tabular}

Das Definieren von Menus erfolgt so:

\def\opt#1(#2){{\rm[}#1\<#2>{\rm]}}%
{\obeylines\tt
m$M$
\opt x(x-Koordinate)
\opt y(y-Koordinate)
\opt v(Vordergrundfarbe)
\opt h(Hintergrundfarbe)
\opt t(Titel)
\opt r(Rahmentyp)
\{
\<Text-Zeilen>
$|$
\<Befehls-Zeilen>
\}
}


Dabei sind {\<x-Koordinate>} und  {\<y-Koordinate>} Zahlen in den 
Bereichen 0 bis 78 bzw.{} 0 bis 22 und geben die Position der linken
oberen Ecke des Rahmens an (Eine Null hei\ss t dabei zentrieren
bez\"uglich des Bildschirms), der die {\<Text-Zeilen>} umschlie\ss t.
{\<Vordergrundfarbe>} und {\<Hintergrundfarbe>} sind Zahlen in
dem Bereich von 0 bis 15 bzw. von 0 bis 7 und bedeuten:
$$\begin{tabular}{rp{3cm}rp{3cm}}
0& black      &8& darkgray \\
1& blue       &9& lightblue \\
2& green     &10& lightgreen \\
3& cyan      &11& lightcyan \\ 
4& red       &12& lightred \\
5& magenta   &13& lightmagenta \\
6& brown     &14& yellow \\
7& lightgray &15& white \\
\end{tabular}
$$

\medskip

{\<Titel>} ist eine Zeichenfolge, die zentriert in die Oberkante des
Rahmens geschrieben wird. \<Rahmentyp> ist 0, 1 oder 2 f\"ur die Anzahl
der Striche des Rahmens, voreingestellt ist 2. Den {\<Befehls-Zeilen>} 
ist der n\"achste Abschnitt gewidmet.

\section{Beschreibung der Befehle.}

Ein Befehl ist eine Zeichenfolge bestehend aus dem Zeichen {\it 
Tastencode}, dem Zeichen {\it Befehlscode} und optionalen {\it 
Parametern}. Er tritt nur in der Definition eines Menus oder einer
Funktionstaste auf.

Als {\it Tastencode} sind im Prinzip alle Zeichen des
ASCII-Codes $>4$ zugelassen, man sollte sich aber auf die druckbaren
beschr\"anken, zumal es f\"ur die wichtigsten anderen sogenannte
Ersatzdarstellungen gibt. N\"amlich
\def\x#1{{$\tt\backslash$}#1 &}%
$$
\begin{tabular}{llll}
\x e f\"ur Escape & \x r f\"ur Return \\
\x n f\"ur Newline & \x b f\"ur Backspace \\
\x t f\"ur Tab & \x{$\tt\backslash$} f\"ur den Backslash selbst \\
\end{tabular}
$$
und zwei sogenannte unechte Tastencodes,
$$
\begin{tabular}{ll}
\x i f\"ur jede Nichtfunktionstaste (immer) und \\
\x a falls noch keine echte Taste ausgef\"uhrt wurde. (alternativ). \\
\end{tabular}
$$

Bei Parametern werden grunds\"atzlich die Zeichen \verb|$|$V$ gegen 
den Inhalt der Variablen $V$ ersetzt. Die Zeichenfolge \verb|$$| wird
jedoch gegen das einzelne Zeichen \verb|$| ersetzt. Ersetzung erfolgt
immer nur eine Stufe weit.

Selektoren werden so benutzt, im Beispiel ist es der Selektor Nummer 4:
  \[
	\hbox{\tt \$4\{{\it Text f"ur 0}|{\it Text f"ur 1}|\ldots|{\it
Text f"ur n}\}}.
  \]
Dabei d"urfen die Texte Variable enthalten, aber keine Selektoren.


Hier nun die unterst\"utzten {\it Befehlscode\/}s:

\begin{tabular}{lp{9,85 true cm}}
\tolerance=9900
\io Q[errorlevel](Quit immediate) Sofortiges Beenden des Programms mit
  Returncode \p.\\
\i q(quit) Entferne das oberste Menu.\\
\i {m$M$}Rufe das Menu $M$ auf. Hierarchisch, rekursiv. Vorsicht
  bei Funktionstasten.\\
\i {g$M$}(goto) Entferne das oberste Menu und rufe dann das Menu 
  $M$ auf.\\
\io S[filename]Sichere Variable und deren Picklisten nach \p.
  Voreinstellung f\"ur \p\ ist der 
  {\it Menu-name} mit Extension {\tt .sav}. Das Sichern hat explizit zu
  erfolgen, dieses wird nicht automatisch beim Programmende ausgef"uhrt.\\
\io L[filename]Laden von Variablen, Menus, etc. Vgl.{} auch 
  {\tt S}.\\
\ii e(DOS-Kommando)Ausf\"uhren des \p\ mit clear.\\
\ii E(DOS-Kommando)Clear. Ausf\"uhrung. Warten auf Taste.\\
\ii >(DOS-Kommando)Ausf\"uhrung an der eingestellten 
  Cursorposition. Ohne clear.\\
\iii ?V(prompt)Variable $V$ mit Prompt \p\ einlesen.\\
\iii pV(prompt)Variable $V$ mit Eingabeauf{}forderung \p\ und 
  Pickliste einlesen.\\
\ii !(Text)(print) Den \p\ an die festgelegte Cursorposition
  mit festgelegten Farben schreiben.\\
\i c (clear) Bildschirm mit der Hintergrundfarbe l\"oschen.\\
\i r (redraw) Ein Neuzeichenen der Menus ausl\"osen. (ohne clear)\\
\i R (Redraw) Ein Neuzeichenen der Menus ausl\"osen. (mit clear)\\
\i t (Taste) Auf Tastendruck warten.\\
\end{tabular}

\begin{tabular}{lp{9,85 true cm}}
\tolerance=9900
\ii x(x-Koordinate)Die \p\ f"ur \verb|!| und \verb|>| festlegen.\\
\ii y(y-Koordinate)Die \p\ f"ur \verb|!| und \verb|>| festlegen.\\
\ii v(Vordergrundfarbe)Die \p\ f"ur \verb|!| festlegen.\\
\ii h(Hintergrundfarbe)Die \p\ f"ur \verb|!| festlegen.\\
\iii \%V(Pickliste)Die Pickliste der Variablen $V$ auf den Wert
  \p\ setzen.\\
\iii \$V(Text)Den Wert der Variablen $V$ auf \p\ setzen.\\
\iii \$N(Wert)Den Wert des Selektors $N$ auf \p\ setzen.\\
\ik iN (invert) Wahrheitswert des Selektors $N$ negieren.\\
\iko +N[number] (increment) Den Selektor $N$ inkrementieren (bis h"ochstens
\p. Ist \p\ negativ so wird bei "Uberschreitung von $-$\p\ auf 0 zur"uckgestellt).\\
\iko -N[number] (decrement) Den Selektor $N$ dekrementieren (bis
h"ochstens \p. Ist \p\ negativ so wird bei Unterschreitung von 0 auf
$-$\p\ zur"uckgestellt).\\
\iii TN(Argumente) Das Programm {\tt test.exe} mit \p\ als Argumente
aufrufen, und dessen Ergebnis (errorlevel) an den Selektor $N$ binden.\\
\end{tabular}

Ist die Environmentvariable {\tt TESTPRG} gesetzt, so wird statt des
Programms {\tt test.exe} das dort bezeichnete Programm benutzt.
Dieses Programm ist in der Lage, logische Verkn"upfungen und elementare
Rechenoperationen auszuf"uhren, die dann als Errorlevel zur"uckgegeben
werden. Dies ist aber in dessen Beschreibung erk"l"art.




\section{"Anderungen.}
\subsection{Version 1.4}

Selektoren implementiert. Das sind Variable mit den Ziffern {\tt
0}--{\tt 9} als Namen. Diese arbeiten intensiv mit dem Programm {\tt
test.exe} zusammen.

\subsection{Version 1.5}

Selektoren haben jetzt einstellbare obere Schranken. ``This product
uses the SPAWNO routines by Ralf Brown to minimize memory use while
shelling to DOS and running other programs.'' Das Menuprogramm wird also
ins XMS verlagert, wenn der Platz knapp ist.



\section{Shareware.}

Shareware ist ein f\"ur den Autor einfaches Vertriebssystem und auch
f\"ur den Anwender ein vorteilhaftes Konzept. Er kann die Software
beliebig lange testen und an $x$-beliebige Personen weitergeben, solange
das ganze Paket beisammen bleibt. Das Paket darf jedoch nicht f\"ur Geld
verkauft werden, da sich der Autor sein Copyright vorbeh"alt.
Falls sich der Anwender entschlie"st, das Programm regelm"a"sig zu
nutzen so wird ein geringer Geldbetrag f"allig, der als Honorar f"ur
die geleistete Entwicklungsarbeit anzusehen ist und einen Anreiz
darstellt, weitere Programme zu schreiben.

Dieses Paket "`Menu"', bestehend aus {\tt menu.exe}, {\tt menu.c}, 
{\tt menu.TeX}, {\tt menu.dvi} und den Beispielen {\bf ist} Shareware.
Bei regelm"a"siger Benutzung schicken Sie daher bitte 20DM an den Autor
unter folgender Adresse:

\medskip
\centerline{\vbox{\hbox{Martin Lercher}\hbox{Schl\"o\ss lgartenweg 40}%
\hbox{D-8415 Nittenau}}}
\smallskip

email: {\tt rrzc1.rz.uni-regensburg.de!c3524}

\medskip

{\bf Haftungsausschlu"s:}
Wie "ublich "ubernehme ich keine Haftung f"ur irgenwelche mittelbar oder
unmittelbar durch das Paket entstandene Sch"aden. Au"serdem garantiere
ich nicht f"ur die universelle Verwendbarkeit des Programms. Ich garantiere
aber, da"s ich es nach bestem Wissen und Gewissen getestet und f"ur
fehlerfrei befunden habe. Kurz und gut: Alles, was Sie mit dem Programm
anstellen geschieht auf Ihre eigene Gefahr.

Sollten Sie also eine unersetzliche 20MB Datei mit Namen {\tt menu.sav}
in Ihrem aktuellen Verzeichnis haben und starten dann {\tt menu.exe},
dann wundern Sie sich nicht, wenn diese aufgrund eines Savebefehls
\verb|S| hinterher nur noch 65 Bytes lang ist\dots

\newpage

\twocolumn\section{Beispiele.}
\baselineskip = 14.4pt plus 5pt minus 1pt

\subsection{Ein simples \TeX-Menu}

\begin{verbatim}
!TeX-Menu vom 13.6.91.
!Fragen bitte an M. Lercher.
ma
tHauptmenue
y5
{
Fuer $n
Datei ist: $a
Format: $F
-------------------------------
e - edit    = - Datei festlegen
t - TeX     d - dir *.tex
v - view    P - Pfad Wechseln
D - Drucken F - TeX-Format
B - BEENDEN RET - MSDOS
|
eeedwin $a.log $a.TeX
tetex $f $a
veview $a
Dedrtex $a
=c
=y5
=v2
=!Verzeichnis der TeX-Dateien:
=>dir *.tex /w
=paTeX-Datei ohne Extension:
FpFTeX-Format (mit &) eingeben:
PmP
\rpdMSDOS:
\rE$d
dEdir *.tex /w
BS
Bq
}
\end{verbatim}

\begin{verbatim}
mP
tWechseln
y17
{
ESC - ENDE
j - Janich
J - Janich, AUFG
k - Knorr
n - Neukirch
|
\eecd \$N
\eq
\iS
j$NJanich
J$NJanich\Aufg
k$NKnorr
n$NNeukirch
\iecd \$N
\iL
\iq
}
\end{verbatim}

\begin{verbatim}
$n(???)
$f&amsplain
\end{verbatim}





\subsection{Ein besseres \TeX\ und Metafont-Menu}
\begin{verbatim}
!TeX-Menue vom 5.7.91. 
!Fragen an Martin Lercher.

:F1-TeX F2-MF F9-Del F10-DOS>

1ga
2gb
9mc
0mD

mC
r1
x1
tAufraeumen
{
l - del *.log
a - del *.aux
b - del *.bak
d - del *.dlg
A - alles
|
Aedel *.log
A>del *.aux
A>del *.bak
A>del *.dlg
ledel *.log
aedel *.aux
bedel *.bak
dedel *.dlg
\iq
}
\end{verbatim}

\begin{verbatim}
mD
tMSDOS
y19
{
d   - dir *.TeX /w
D   - dir *.* /w
RET - Dos-Kommando
|
dEdir *.tex /w /p
DEdir /w /p
\rpDMSDOS: 
\rE$d
\iq
}
\end{verbatim}

\begin{verbatim}
ma
tTeX
{
W - Workfile $b.TeX
M - Mainfile $a.TeX
F - Format   $f
O - DrOption $o
-------------------------
e - edit     t - TeX
v - view     D - Drucken
E - Edit ... Q - Beenden.
|
Ec
Ey5
Ev2
E!Verzeichnis:
E>dir /w/p
EpEEdit: 
Eeue $E
eeue $b.TeX $a.log
tetex $f $a
veview $a /+
Dedrtex $a $o
Wc
Wy5
Wv2
W!Verzeichnis der TeX-Dateien: 
W>dir *.tex /w
WpbWorkfile ohne Extension:
Mc
My5
Mv2
M!Verzeichnis der TeX-Dateien: 
M>dir *.tex /w
MpaMainfile ohne Extension:
F%F&plain
FpFTeX-Format (mit &) eingeben:
O?oDruck Optionen: 
QS
Qq
}
\end{verbatim}

\begin{verbatim}
mB
tMetafont
{
W - Workfile $w.mf
J - Jobfile  $j.mfj
M - MF-file  $x.mf
B - Base     $y
-------------------------
e - edit     m - Metafont
j - MFJob
E - Edit ... Q - Beenden.
|
eeue $w.mf $x.log
jemfjob $j
memf $y $x.mf
Ec
Ey5
Ev2
E!Verzeichnis:
E>dir /w/p
EpEEdit: 
Eeue $E
Wc
Wy5
Wv2
W!Verzeichnis der MF-Dateien: 
W>dir *.mf /w
WpwWorkfile ohne Extension:
Mc
My5
Mv2
M!Verzeichnis der MF-Dateien: 
M>dir *.mf /w
MpxMF-file ohne Extension:
Jc
Jy5
Jv2
J!Verzeichnis der MFJ-Dateien: 
J>dir *.mfj /w
JpjJobfile ohne Extension:
B%y&plain
BpyMF-Format (mit &) eingeben:
QS
Qq
\aemode co80
}
\end{verbatim}

\begin{verbatim}
$y&plain
$f&plain
\end{verbatim}




\subsection{Ein Schmarrnmenue.}
\begin{verbatim}
!Ein Menue-Programm vom 11.6.91
!Autor: Martin Lercher.
:Bitte waehlen Sie aus:

$AHallo.
$BNoch was?

mA
t$A
{
Das ist das Hauptmenue.
Waehle aus:
(d)ir (m)apmem (1)Konfig
(x)Exit (RET) Dos
|
dedir
mEmapmem
1mB
xS
xq
llDas ist ein Fehler.
\r?CDOS:
\rE$C
}
\end{verbatim}

\begin{verbatim}
mB
y4
tAnderes Menue
{
$$B = $B
Lade $$A,  Lade $$B
q - zurueck. 2 - print
ESC $$A = abcdef
|
a?aWas ist $$A:
bpbWas ist $$B:
qq
\e$Aabcdef
2mC
}
\end{verbatim}

\begin{verbatim}
0mX
1edir *.c
\end{verbatim}

\begin{verbatim}
# Das exit Menu
mX
v3 x2 r1
y20 {
Exit (mit sichern) (J/N/S)
|
jQ
nq
sS
sQ
}
\end{verbatim}

\begin{verbatim}
mc
{
q - quit
a - a ausgeben
b - b ausgeben
c - c ausgeben
0 - loeschen
w - Warnung
|
qq
0R
ax1
ay4
a!a ist $a
bv3
bh14
b!b ist $b
ch2
cc
c!----------
c!c ist $c
c!----------
ct
cr
wh2
wc
wv1
wx30
wy12
w!Keine Experimente!
wt
wh0
wc
wr
}
\end{verbatim}

\centerline{-- FIN --}


\end{document}
